\documentclass[a4paper, 12pt]{article}
\usepackage[utf8]{inputenc}
\usepackage[english, ukrainian]{babel}
\usepackage{amsmath, amssymb}
\usepackage[top = 2 cm, left = 1 cm, right = 1 cm, bottom = 2 cm]{geometry} 

\newenvironment{system}{\begin{equation} \left\{\begin{aligned}}{\end{aligned} \right. \end{equation}}
\newenvironment{system*}{\begin{equation*} \left\{\begin{aligned}}{\end{aligned} \right. \end{equation*}}

\usepackage{fancyhdr}
\pagestyle{fancy}
\rhead{Нікіта Скибицький, ОМ-3}
\cfoot{\thepage}

\usepackage{multicol}
\usepackage{graphicx}
\usepackage{float}

\usepackage{amsthm}
\theoremstyle{definition}
\newtheorem{theorem}{Теорема}[subsection]
\newtheorem{definition}{Визначення}
\newtheorem{prove}{Доведення}
\newtheorem{problem}{\normalfont{\textit{Задача}}}[section]
\newtheorem*{solution}{Розв'язок}
\newtheorem*{side_comment}{Зауваження}

\allowdisplaybreaks
\setlength\parindent{0pt}

\newcommand{\argmax}{\arg\max}
\newcommand{\argmin}{\arg\min}

\newcommand{\dif}{\mathrm{d}}
\newcommand{\dydx}{\dfrac{\dif y}{\dif x}}
\newcommand{\dxdt}{\dfrac{\dif x}{\dif t}}
\newcommand{\dydt}{\dfrac{\dif y}{\dif t}}

\newcommand{\partialDerivative}[2]{\dfrac{\partial #1}{\partial #2}}

\newcommand{\NN}{\mathbb{N}} 
\newcommand{\ZZ}{\mathbb{Z}}
\newcommand{\QQ}{\mathbb{Q}}
\newcommand{\RR}{\mathbb{R}}
\newcommand{\CC}{\mathbb{C}}

\DeclareSymbolFont{extraitalic}{U}{zavm}{m}{it}
\DeclareMathSymbol{\stigma}{\mathord}{extraitalic}{168}

% \renewcommand{\Re}{\text{Re}}
% \renewcommand{\Im}{\text{Im}}

\newcommand{\const}{\text{const}}

\newcommand{\LaReF}[1]{(\ref{#1})}

\renewcommand{\epsilon}{\varepsilon}
\renewcommand{\phi}{\varphi}

\newcommand{\ndiv}{\!\!\not|\,}
\newcommand{\nequiv}{\not\equiv}

\newcommand{\ws}{\text{ }}

\newcommand{\Max}{\displaystyle\max\limits}
\newcommand{\Min}{\displaystyle\min\limits}
\newcommand{\Sum}{\displaystyle\sum\limits}
\newcommand{\Int}{\displaystyle\int\limits}
\newcommand{\Lim}{\displaystyle\lim\limits}
\newcommand{\Prod}{\displaystyle\prod\limits}

\allowdisplaybreaks

\newcommand{\underbracetext}[2]{\underset{#1}{\underbrace{#2}}}
\newcommand{\underbrackettext}[2]{\underset{#1}{\underbracket{#2}}}

\numberwithin{equation}{subsection}

\DeclareMathOperator{\sech}{sech}
\DeclareMathOperator{\csch}{csch}

\DeclareMathOperator{\arcsec}{arcsec}
\DeclareMathOperator{\arccot}{arccot}
\DeclareMathOperator{\arccsc}{arccsc}

\DeclareMathOperator{\arccosh}{arccosh}
\DeclareMathOperator{\arcsinh}{arcsinh}
\DeclareMathOperator{\arctanh}{arctanh}
\DeclareMathOperator{\arcsech}{arcsech}
\DeclareMathOperator{\arccsch}{arccsch}
\DeclareMathOperator{\arccoth}{arccoth}

\DeclareMathOperator{\Arcsinh}{Arcsinh}
\DeclareMathOperator{\Arccosh}{Arccosh}
\DeclareMathOperator{\Arctanh}{Arctanh}
\DeclareMathOperator{\Arccoth}{Arccoth}

\DeclareMathOperator{\Arccos}{Arccos}
\DeclareMathOperator{\Arcsin}{Arcsin}
\DeclareMathOperator{\Arctan}{Arctan}
\DeclareMathOperator{\Arccot}{Arccot}

\DeclareMathOperator{\Arg}{Arg}

\DeclareMathOperator{\Real}{Re}
\DeclareMathOperator{\Imag}{Im}

\DeclareMathOperator{\Ln}{Ln}

\DeclareMathOperator{\signum}{sgn}


\lhead{Лекції з аналізу даних, 2018}

\begin{document}

Під час аналізу даних виділяються наступні етапи: отримання вхідної інформації, безпосередньо сама обробка її, аналіз та інтерпретація результатів обробки даних. \\

Головне зробити правильні висновки з результатів. \\

Значення змінних які спостерігаються можуть бути як \textit{кількісні} так і \textit{якісні}. Якісні змінні поділяють на \textit{ординальні} та \textit{номінальні}. Ординальні змінні називають \textit{порядковими}, а номінальні -- \textit{класифікаційними}. Обидва типи змінних приймають свої значення з деякої множини, елементи якої називають \textit{градаціями}. Градації, які приймає як свої значення ординальна змінна, природно \textbf{впорядковані за степенем прояву властивості}. Градації номінальної змінної такого порядку \textbf{на мають}. Серед якісних змінних виділяють \textit{категоризовані} та \textit{не категоризовані}. \\

До категоризованих змінних відносять змінні, для яких повністю визначена множина градацій та правило віднесення значення змінної, яке спостерігається, до певної градації. \\

Змінні ще поділяють на \textit{дискретні} та \textit{неперервні}.

\section{Групування змінних}

$\xi$ -- скалярна змінна, яка досліджується. \\

Вибірка об'єму $n$: $x_1, x_2, \ldots, x_n$. \\

У випадку великих об'ємів вибірок виникає бажання провести деяке перетворення їх з метою стиснення даних без суттєвої втрати вибірками інформативності, а тільки згодом проводити обробку цих перетворених даних. Як правило, його застосовують при обробці спостережень над неперервними змінними, коли об'єм вибірки перевищує 50, а над дискретними змінними, коли кількість значень $m$, які вони приймають, перевищує 10. \\

Перехід до згрупованих даних:
\begin{enumerate}
    \item Визначити $x_{\min} = \Min_i (x_i)$, $x_{\max} = \Max_i (x_i)$;
    
    \item Інтервал $[x_{\min}, x_{\max}]$ розбивають на $s$ однакових під-інтервалів $[a_i, b_i)$, $i = \overline{1,s}$. Зазвичай $5 \le s \le 30$. Зазвичай $s = 1 + [\log_2 n]$ або $s = [10 \log_{10} (n)]$;
    
    \item $x_i^* = \dfrac{a_i + b_I}{2}$ -- центральна точка. \\
    
    $v_i$ -- кількість вимірів з вибірки що належать інтервалу $[a_i, b_i)$. \\
    
    $\{x_1, x_2, \ldots, x_n\} \mapsto \{x_i^*, v_i\}_{i=1}^s$ $\left( \Sum_{i=1}^s v_i = n \right)$. \\
    
    Рекомендується $v_i \ge 5$, в разі $v_i < 5$ сусідні інтервали зливаються в один.
\end{enumerate}

\textbf{Зауваження!} При проведенні групування даних зовсім не обов'язково брати під-інтервали однакової довжини.

$F_\xi (x) = P\{ \xi < x\}$ -- функція розподілу, $p_\xi(x)$ -- функція щільності, $\{ y_i, p_i \}_{i=1}^m$ -- полігон ймовірності, якщо $\xi$ -- дискретна випадкова величина, що набуває значення $y_i$ з ймовірністю $p_i$, $i=\overline{1,m}$. \\

Оцінка характеристик по згрупованим даним:

\end{document}
