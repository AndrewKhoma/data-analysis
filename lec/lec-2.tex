\subsubsection{Імовірнісний графік}
Ідея та ж сама. Зі спотвореною віссю $у$. Маємо множину $\{x \in \RR, y \in [0,1]\}$, яку розтягують за правилом $(x, y) \mapsto \left(x, F_0^{-1}(y)\right)$, де $y = \widehat{F}_\xi (x)$. \\

Папір, де спотворюється масштаб називається \textit{імовірнісним папером}. \\

Якщо в якості розподілу взяти нормальний розподіл, то такий папір називається \textit{нормальним імовірнісним папером}. \\

Будуємо графік функції $y = F_\xi(x)$ -- для спостереження величини $\xi$.
\begin{enumerate}
	\item У випадку, коли $H_0: F_\xi(\cdot)\in \mathcal{F}$, то отримаємо майже пряму.
	\item Виявляємо наявність \textit{аномальних спостережень}: якщо маємо точки, що лежать осторонь, то перевіряємо їх на аномальність.
\end{enumerate}
\subsubsection{Висячі гістобари}
Використовується для перевірки нормальності вибірки. Нехай по вибірці $\xi$: $x_1$, $\ldots$, $x_n$ підраховано мат. сподівання $\bar x(n)$ та вибіркова дисперсія $s^2(n)$. \\

\textit{Найбільш узгодженим нормальним розподілом} для спостережень за $\xi$ будемо називати такий нормальний розподіл $N\left(x(n), s^2(n)\right)$. \\

Спочатку будуємо графік щільності з вибірки $\xi: x_1, \ldots, x_n$. \\

В центрах групування даних до графіка підвішуються прямокутні гістобари, довжина яких пропорційна відносній частоті потрапляння у відповідний інтервал групування. Якщо основа
цих гістобар не суттєво відхиляється від осі $Ох$ -- гіпотеза про нормальність вибірки приймається. % figure 3
\subsubsection{Підвішена коренеграма}
Для кожного інтервалу групування даних визначають $v_e$ -- емпірична частота потрапляння в інтервал, а також теоретичне значення частоти $v_T$ згідно гіпотези про найбільш узгоджений нормальний розподіл. Потім на графіку відкладають такі різниці: $\sqrt{v_e}-\sqrt{v_T}$. І якщо ці значення не значно відхиляються від нуля, то гіпотеза про нормальність вибірки приймається. % figure 4
\subsection{Спостереження за двома змінними}
Використовуються:
\begin{enumerate}
	\item Діаграма розсіювання.
	\item Таблиця спряженості.
\end{enumerate}
\subsubsection{Діаграми розсіювання}
Маємо дві вибірки $\xi: x_1, \ldots, x_n$ та $\eta: y_1, \ldots, y_n$. Використовують для з'ясування класу залежності між парою кількісних змінних, а також для з'ясування наявності аномальних спостережень у вибірці. % figure 5
\subsubsection{Таблиця спряженості}
Використовуються для представлення спостережень над номінальними, ординальними, кількісними дискретними (скінченими), кількісними неперервними (згрупованими змінними). \\

Нехай є змінна $\xi$ яка має $r_1$ градацій, та змінна $\zeta$ яка має $r_2$ градацій:
\begin{table}[H]
	\centering
	\begin{tabular}{|c|c|c|c|c|c|}
	\hline
	& $1$ & $2$ & $\ldots$ & $r_1$ & $\sum$ \\
 \hline
	$1$ & $n_{11}$ & $n_{12}$ & $\ldots$ & $n_{1r_1}$ & $n_{1*}$ \\
 \hline
	$2$ & $n_{21}$ & $n_{22}$ & $\ldots$ & $n_{2r_1}$ & $n_{2*}$ \\
 \hline
	$\vdots$ & $\vdots$ & $\vdots$ & $\ddots$ & $\vdots$ & $\vdots$ \\
 \hline
	$r_2$ & $n_{r_21}$ & $n_{r_22}$ & $\ldots$ & $n_{r_2r_1}$ & $n_{r_2*}$ \\
 \hline
	$\sum$ & $n_{*1}$ & $n_{*2}$ & $\ldots$ & $n_{*r_1}$ & $n_{**}$ \\
 \hline
	\end{tabular}
\end{table}
Де $n_{ij}$ -- кількість таких спостережень $\{\xi=i,\zeta=j\}$, позначимо $n_{i*} = \Sum_{j=1}^{r_1} n_{ij}$, $n_{*j} = \Sum_{i=1}^{r_2} n_{ij}$. 
\section{Попередня обробка}
До попередньої обробки відносять:
\begin{itemize}
	\item розвідувальний аналіз;
	\item обчислення основних характеристик спостережуваних величин;
	\item видалення аномалій;
	\item перевірка основних гіпотез;
	\item перевірка на стохастичність вибірки.
\end{itemize}
\subsection{Характеристики випадкових (скалярних) величин}
\subsubsection{Квантилі та процентні точки}
\textit{Квантилем рівня} $0 < q < 1$ для \textit{неперервної} випадкової величини $\xi: F_\xi(\cdot)$ називається значення $u_q(F): P\{ \xi < u_q(F)\}=q$. \\

\textit{Квантилем рівня} $0 < q < 1$ для \textit{дискретної} випадкової величини $\xi: F_\xi(\cdot)$ називається будь-яке значення $U_q(F)\in (y_{i(q)}, y_{i(q)+1}]$, для границь якого виконується $P\{ \xi < y_{i(q)}\} < q$ та $P\{ \xi < y_{i(q) + 1}\} \ge q$. \\

Вибіркові квантилі $\widehat{u}_q(F)$ визначаються як квантилі відповідних емпіричних розподілів. \\

\textit{Q-процентною точкою} ($0 < Q <100$) для \textit{неперервної} випадкової величини $\xi: F_\xi(\cdot)$ називається значення $\omega_Q(F): P\{ \xi \ge \omega_Q(F) \} = \frac{Q}{100}$. \\

\textit{Q-процентною точкою} ($0 < Q <100$) для \textit{дискретної} випадкової величини $\xi: F_\xi(\cdot)$ називається довільне значення $w_Q(F) \in (y_{i(Q)}, y_{i(Q)+1}]$, для границь якого виконується $P\{ \xi \ge y_{i(Q)}\} > \frac{Q}{100}$ та $P\{ \xi \ge y_{i(Q)+1}\} \le \frac{Q}{100}$. \\

Квантиль та процентна точка пов'язані певним співвідношенням, а саме \[\omega_Q(F)=u_{1-\frac{Q}{100}}(F)\quad\text{та} \quad u_q(F) = \omega_{(1-q)100}(F). \]
Введемо додаткові характеристики розподілу, похідні від перших двох. \\

\textbf{Приклади квантилей:}
\begin{enumerate}
	\item \textit{Медіаною} називається квантиль рівня $0.5$: $u_{0.5}$.
	\item $u_{0.75}, u_{0.25}$ -- \textit{верхній та нижній квартилі} відповідно.
	\item Значення $\{ u_{\frac{i}{10}}\}_{i=1}^9$ називаються \textit{децилями}.
	\item $\{ u_{\frac{i}{100}}\}_{i=1}^{99}$ -- \textit{процентилі}.
	\item \textit{Інтерквантильною широтою рівня} $p: 0 < p < \frac{1}{2}$ називається величина $u_{1-p}-u_p$.
	\item \textit{Інтерквартильною широтою} називається величина $u_{0.75} - u_{0.25}$ тобто $p = 0.25$ 
	\item Половина інтерквантильної широти називається \textit{імовірнісним відхиленням}.
\end{enumerate}
\subsubsection{Характеристики положення центру значень}
\begin{enumerate}
	\item \textit{Математичне сподівання} $M\xi$, та його вибірковий аналог \[\bar{x}(n) = \dfrac{1}{n}\Sum_{i=1}^n \xi_i.\]
	\item \textit{Геометричне середнє} $G_\xi = e^{M \ln \xi}$ для $\xi: P\{\xi \le 0\} = 0$. \[\widehat{G}_\xi(n) = \sqrt[n]{\Prod_{i=1}^n x_i}.\]
	\item \textit{Середнє гармонічне} $H = M^{-1}\left(\dfrac{1}{\xi}\right)$ для $\xi: P\{\xi \le 0\} = 0$. \[\widehat{H}_\xi = \left( \dfrac{1}{n} \Sum_{i=1}^n \dfrac{1}{x_i}\right)^{-1}.\]
	\item \textit{Мода} $x_{\text{mod}}=\argmax_a P\{\xi = a\}$ для дискретних випадкових величин (визначається за гістограмою) та $x_{\text{mod}}=\argmax_x f(x)$ в неперервному випадку.
	\item \textit{Медіаною} називається $x_\text{med} = u_{0.5}$.
\end{enumerate}