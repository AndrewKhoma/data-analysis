\subsection{Рангові критерії однорідності}

Розглянемо випадкові величини $\xi_1, \xi_2, \ldots, \xi_k$ з функціями розподілу $F_1(x)$, $F_2(x)$, $\ldots$, $F_k(x)$. На їх основі сформуємо об'єднану вибірку $\nu_1, \nu_2, \ldots, \nu_n$, а для кожної змінної $\xi_i$ отримаємо незалежні спостереження $x_1^{(i)}, x_2^{(i)}, \ldots, x_{n_i}^{(i)}$, $i = \overline{1,k}$. Тоді сформована вибірка буде об'ємом $n = \sum_{i=1}^k n_i$. Для спрощення вважаємо, що всі виміри $\nu_i$, $i=\overline{1,n}$ різні. Розташувавши ці значення у порядку зростання, отримаємо варіаційний ряд $\nu_{(1)}, \nu_{(2)}, \ldots, \nu_{(n)}$. Члени варіаційного ряду називають порядковими статистиками. \\

Рангом спостереження $\nu_i$ ($i = \overline{1,n}$) називається його порядковий номер у побудованому варіаційному ряді, позначається $R_{i,n}$ -- ранг спостереження $\nu_i$ ($i = \overline{1,n}$).

\subsubsection{Статистика для лінійного рангового критерію}

\[ K_i = \Sum_{j = N_i - n_i + 1}^{N_i} \phi(R_{j,n}), \quad N_i = \Sum_{j=1}^i n_j, \quad i = \overline{1,k} \]

$K_i$ -- статистика по спостереження над $\xi_i$, $\phi(R_{i,n})$ -- мітка. \\

Потрібно перевірити гіпотезу $H_0: F_1(x) = F_2(x) = \ldots = F_k(x)$, $\forall x$ з рівнем значимості $\alpha$ ($0 < \alpha < 1$). Хочемо переконатись зо всі випадкові величини однаково розподілені.

\subsubsection{Випадок двох вибірок}

Гіпотеза $H_0: F_1(x) = F_2(x)$, $\forall x$ з рівнем значимості $\alpha$ ($0 < \alpha < 1$). Альтернативні гіпотези:
\begin{align*}
    H_{11}: & F_1(x) = F_2(x - \Delta), \forall x, (\Delta \ne 0) \\
    H_{12}: & F_1(x) = F_2(x - \Delta), \forall x, (\Delta > 0) \\
    H_{13}: & F_1(x) = F_2(x - \Delta), \forall x, (\Delta < 0)
\end{align*}
Всі критерії розглядаються над першою змінною $\xi_1$. \\

\textbf{Критерій нормальних міток (Фішера)} \\

$C = \sum_{i=1}^{n_1} M(R_{i, n}, n)$, де $M(m, n)$ -- математичне сподівання $m$-ої порядкової статистики вибірки довжини $n = n_1 + n_2$ нормально розподіленої величини з параметрами $0$ та $1$. \\

Статистика $C$ має наступні характеристики при справедливості нульової гіпотези:
\[ M C = 0, D C = \dfrac{n_1 n_2}{n(n-1)} \Sum_{i=1}^n (M(i, n))^2 \]

\textbf{Критерій Ван дер Вардена} \\

Статистика критерію має вигляд $ V = \sum_{i=1}^{n_1} \Phi^{-1} \left( \frac{R_{i,n}}{n+1} \right)$, де $\Phi^{-1}(x)$ -- функція обернена до функції розподілу з параметрами $0$ та $1$, причому коли справедлива нульова гіпотеза, то
\[ M V = 0, DV = \dfrac{n_1 n_2}{n(n-1)} \Sum_{i=1}^n \left( \Phi^{-1} \left( \dfrac{i}{n+1} \right) \right)^2 \]

\textbf{Критерій Вілкоксона} \\

Статистика критерію має вигляд $S = \sum_{i=1}^{n_1} R_{i,n}$, причому коли справедлива нульова гіпотеза, то
\[ MS = \dfrac{n_1(n + 1)}{2}, \quad DS = \dfrac{n_1 n_2 n}{12} \]

Процедура використання статистик $C$, $V$, $S$ для перевірки гіпотези $H_0$ однакова: позначимо через $U$ деяку статистику ($C$, $V$, або $S$), $\bar{U} = \frac{U - M U}{\sqrt{D U}}$. В залежності від альтернативної гіпотези $H_0$ приймається якщо:
\begin{itemize}
    \item $|\bar{U}| < u_{1 - \alpha / 2}$, якщо альтернатива $H_{11}$,
    
    \item $\bar{U} < u_{1 - \alpha}$, якщо альтернатива $H_{12}$,
    
    \item $\bar{U} > u_\alpha$, якщо альтернатива $H_{13}$,
\end{itemize}
де $u_\alpha$ -- квантиль рівня $\alpha$ для нормального розподілу з параметрами $0$ та $1$. \\

По мірі спадання потужності критерії розташовуються так: критерій нормальних міток Фішера, критерій Ван дер Вардена, критерій Вілкоксона.

\subsubsection{Загальний випадок}

Гіпотеза $H_0$: $F_1(x) = F_2(x) = \ldots = F_k(x)$, $\forall x$ з рівнем значимості $\alpha$ ($0 < \alpha < 1$). 

\begin{enumerate}
    \item Будуємо об'єднану вибірку $\nu_1, \nu_2, \ldots, \nu_n$ об'єму $n = \sum_{i=1}^k n_i$, а потім відповідний варіаційний ряд $\nu_{(1)}, \nu_{(2)}, \ldots, \nu_{(n)}$.
    
    \item Для кожної $\xi_i$ підрахуємо статистику $K_i = \sum_{j = N_i - n_I + 1}^{N_i} \psi(R_{j, n})$. 
    
    Для неї підійде будь-яка статистика з попередніх критеріїв:
    \begin{multline*} 
        C_i = \Sum_{j = N_i - n_i + 1}^{N_i} M(R_{j, n}, n), \\
        V_i = \Sum_{j = N_i - n_i + 1}^{N_i} \Phi^{-1} \left( \dfrac{R_{i,n}}{n+1} \right), \\
        S_i = \Sum_{j = N_i - n_i + 1}^{N_i} R_{j, n}.
    \end{multline*}
    
    \item Далі знаходимо їх стандартизовані значення \[\bar{K}_i = \dfrac{K_i - MK_i}{\sqrt{DK_i}}.\]
    
    \item Тепер рахуємо статистику \[X^2 = \Sum_{i=1}^k \bar{K}_i^2.\] 
\end{enumerate}

Нульова гіпотеза приймається якщо $X^2 < \chi_\alpha^2(k-1)$, де $\chi_\alpha^2(k)$ -- $\alpha \cdot 100\%$ процентна точка $\chi^2$-розподілу з $k$ степенями свободи.

\subsubsection{Перевірка симетрій розподілу ранговими критеріями}

Маємо ряд незалежних спостережень $x_1, x_2, \ldots, x_n$ над випадковою величиною $\xi$ з функцією розподілу $F(x)$. Перевіримо симетричність розподілу відносно точки $x_0$. \\

Гіпотеза: для дискретної випадкової величини $H_0: F(x_0 + x) = 1 - F(x_0 - x + 0)$, $\forall x$, або ж для неперервної випадкової величини $H_0: p(x_0 + x) = p(x_0 - x)$, $\forall x$. \\

Перевірка проводиться з деяким рівнем значимості $\alpha$ ($0 < \alpha < 1$). \\

Побудуємо послідовність $z_1, z_2, \ldots, z_n$, де $z_i = |x_i - x_0|$, $i = \overline{1, n}$, а далі сформуємо варіаційний ряд $z_{(1)}, z_{(2)}, \ldots, z_{(n)}$. \\

Абсолютним рангом виміру $x_i$ називається порядковий номер значення $x_i = |x_i - x_0|$ у варіаційному ряді $z_{(1)}, z_{(2)}, \ldots, z_{(n)}$, позначатимемо його як $R_{i,n}^+$ ($i=\overline{1,n}$). \\

Розіб'ємо вибірку $x_1, x_2, \ldots, x_n$ на дві вибірки, в першій всі виміри більше $x_0$. в іншій решта. Позначення для індексів з першої вибірки $I^+ = \{ i : x_i > x_0, i = \overline{1,n} \}$. Тепер можна порівняти дві наші вибірки на однорідність. \\

\textbf{Аналог критерію нормальних міток} \\

Статистика критерію має вигляд $C^+ = \sum_{i \in I^+} M^+(R_{i,n}^+, n)$, при справедливості $H_0$: \[ MC^+ = \dfrac{n}{\sqrt{2\pi}}, \quad DC = \dfrac{1}{4} \Sum_{i=1}^n (M^+(i, n))^2. \]

\textbf{Аналог критерію Ван дер Вардена} \\

Статистика критерію має вигляд $V^+ = \sum_{i \in I^+} \Phi^{-1} \left( \frac{1}{2} + \frac{R_{i,n}^+}{2(n + 1)} \right)$, при справедливості $H_0$: \[ MV^+ = \dfrac{1}{2} \Sum_{i=1}^n \Phi^{-1} \left( \dfrac{1}{2} + \dfrac{i}{2(n + 1)} \right), \quad DV^+ = \dfrac{1}{4} \Sum_{i=1}^n \left( \Phi^{-1} \left( \dfrac{1}{2} + \dfrac{i}{2(n + 1)} \right)^2 \right) \]

\textbf{Аналог критерію Вілкоксона} \\

Статистика критерію має вигляд  $S^+ = \sum_{i\in I^+} R_{i, n}^+$, при справедливості $H_0$: \[ MS^+ = \dfrac{(n+1)n}{4}, \quad DS^+ = \dfrac{n(n+1)(2n+1)}{24}. \]

Нехай $U^+$ -- одна з вищенаведених статистик. Стандартизуємо $U^+$: \[\bar{U}^+ = \frac{U^+ - NU^+}{\sqrt{DU^+}}.\] Область прийняття гіпотези $H_0$: $|\bar{U}^+| < u_{1 - \alpha / 2}$, де $u_\alpha$ -- квантиль рівня $\alpha$ нормального розподілу з параметрами $0$ та $1$.

\subsubsection{Визначення рангів у випадку наявності нерозрізнимих значень}

Нехай $\nu_1, \nu_2, \ldots, \nu_n$ -- об'єднана вибірка, побудована на основі спостережень над змінними, що досліджуються. Існує два варіанти однакових спостережень:
\begin{enumerate}
    \item спостереження стосуються однієї змінних, тоді використовується метод випадкового рангу: ранг однакових елементів є довільним числом, яке припало на цю множину значень.
    
    \item спостереження стосуються різних змінних, тоді використовують або вже відомий нам метод випадкового рангу, або метод середньої мітки: всім рівним спостереженням присвоюють середнє значення мітки підраховане за множиною рангів, яка відповідає цій групі нерозрізними вимірів.
\end{enumerate}

\textbf{Корекція алгоритмів рангових критеріїв:} \\

$DC = \frac{n_1 n_2}{n(n-1)} \sum_{i=1}^g \tau_i \bar{M}_i^2$ для критерію нормальних міток Фішера, для критерію Ван дер Вардена $DV = \frac{n_1 n_2}{n(n-1)} \sum_{i=1}^g \tau_i \left( \bar{\Phi}_i^{-1} \right)^2$ , де $g$ -- кількість груп нерозрізнимих спостережень, $\tau_i$ -- кількість значень у $i$-ій групі.