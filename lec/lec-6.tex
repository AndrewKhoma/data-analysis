Якщо всі прояви об'єктів різні, то маємо $x^{(0)}$, $x^{(1)}$, $\ldots$, $x^{(q)}$ -- спостереження, $x_*^{(0)}$, $x_*^{(1)}$, $\ldots$, $x_*^{(q)}$ -- ранжировка. \\

При наявності по деякій зміні групи об'єктів з однаковим проявом досліджуваної властивості, цим об'єктам присвоюють ранг, який дорівнює середньому арифметичному номерів тих місць, які припали на цю групу об'єктів з нерозрізненими рангами. Такий ранг називається зв'язаний (об'єднаний). \\

Будується таблиця рангів для доступу до об'єкта:
\begin{table}[H]
	\centering
	\begin{tabular}{|l|c|c|c|c|}
		\hline
		 & $x^{(1)}$ & $x^{(2)}$ & $\ldots$ & $x^{(q)}$ \\
 \hline
		$1$ & $x_1^{(1)}$ & $x_1^{(2)}$ & $\ldots$ & $x_1^{(q)}$ \\
 \hline
		$2$ & $x_2^{(1)}$ & $x_2^{(2)}$ & $\ldots$ & $x_2^{(q)}$ \\
 \hline
		$\vdots$ & $\vdots$ & $\vdots$ & $\ddots$ & $\vdots$ \\
 \hline
		$n$ & $x_n^{(1)}$ & $x_n^{(2)}$ & $\ldots$ & $x_n^{(q)}$ \\
 \hline
	\end{tabular}
\end{table}
рядки -- об'єкти, стовпчики -- змінні.
\subsubsection{Характеристики парного статистичного зв'язку}
Розглядаємо характеристики $x^{(i)}$, $x^{(j)}$. \\

В якості характеристики парного зв'язку між змінними $x^{(i)}$ та $x^{(j)}$ можемо використати \textit{коефіцієнт Спірмана}, який визначається таким чином: \[ \widehat{\tau}_{ij}^{(s)} = 1 - \dfrac{\left\|x_*^{(i)}-x_*^{(j)}\right\|_2^2}{\dfrac{n^3-n}{6}}. \]
\textbf{Властивості рангу коефіцієнта Спірмана}:
\begin{enumerate}
	\item $-1 \le \widehat{\tau}_{ij}^{(s)} \le 1$;
	\item якщо $\widehat{\tau}_{ij}^{(s)} = 0$, то зв'язок відсутній;
	\item якщо $\widehat{\tau}_{ij}^{(s)} = 1$, то ранжировки по змінним співпадають, $x_*^{(i)} =x_*^{(j)}$;
	\item якщо $\widehat{\tau}_{ij}^{(s)} = -1$, то ранжировки по змінним протилежні, $x_*^{(i)} =x_*^{(j)}$.
\end{enumerate}
Розглянемо випадок наявності \textit{нерозрізнених рангів}. В цьому випадку використовується \textit{модифікований коефіцієнт}. Ранговий коефіцієнт Спірмана обчислюється за формулою:
\[ \widehat{\widehat{\tau}}_{ij}^{(s)} = \dfrac{\dfrac{n^3-n}{6}-\left\|x_*^{(i)}-x_*^{(j)}\right\|_2^2-T^{(i)}-T^{(j)}}{\sqrt{\left(\dfrac{n^3-n}{6}-2T^{(i)}\right)\left(\dfrac{n^3-n}{6}-2T^{(j)}\right)}}, \] де \[ T^{(i)} = \dfrac{1}{12} \Sum_{g=1}^{m^{(i)}} \left(n_g^{(i)}\right)^3 - n_g^{(i)} \] -- \textit{корегуючий коефіцієнт}, $m^{(i)}$ -- кількість груп об'єктів з нерозрізненними рангами по змінній $x^{(i)}$, $n_g^{(i)}$ -- кількість членів у $g$-й групі нерозрізнимих рангів по $i$-й змінній. \\

Коли коефіцієнт приймає проміжне значення, то перевіряємо гіпотезу $H_0: \widehat{\tau}_{ij}^{(s)} = 0$. Якщо об'єм вибірки невеликий, то перевіряємо по таблиці, при $n = 4..10$. Якщо ж $n > 10$, то розглядаємо статистику \[\dfrac{\sqrt{n-2}\widehat{\tau}_{ij}^{(s)}}{\sqrt{1-\left(\widehat{\tau}_{ij}^{(s)}\right)^2}},\] що має $t$-розподіл Стьюдента з $(n - 2)$ степенями свободи. Область прийняття гіпотези: \[\left|\dfrac{\sqrt{n-2}\widehat{\tau}_{ij}^{(s)}}{\sqrt{1-\left(\widehat{\tau}_{ij}^{(s)}\right)^2}}\right| < t_{\alpha/2}(n-2). \]
Розглянемо іншу характеристику: \textit{коефіцієнт Кендала}. \textit{Ранговим коефіцієнтом Кендала} для змінних $x^{(i)}$ та $x^{(j)}$ називається величина \[\widehat{\tau}_{ij}^{(k)} = \dfrac{4v\left(x_*^{(i)},x_*^{(j)}\right)}{n(n-1)},\] де $v\left(x_*^{(i)},x_*^{(j)}\right)$ -- кількість перестановок сусідніх елементів у ранжировці $x_*^{(i)}$, яка приводить її до ражировки $x_*^{(j)}$. \\

\textbf{Властивості рангу коефіцієнта Кендала}:
\begin{enumerate}
	\item $-1 \le \widehat{\tau}_{ij}^{(k)} \le 1$;
	\item якщо $\widehat{\tau}_{ij}^{(k)} = 0$, то зв'язок відсутній;
	\item якщо $\widehat{\tau}_{ij}^{(k)} = 1$, то ранжировки по змінним співпадають, $x_*^{(i)} =x_*^{(j)}$;
	\item якщо $\widehat{\tau}_{ij}^{(k)} = -1$, то ранжировки по змінним протилежні, $x_*^{(i)} =x_*^{(j)}$.
\end{enumerate}
Якщо є наявні нерозрізнені ранжировки, то використовують \textit{модифікований коефіцієнт Кендала}: \[ \widehat{\widehat{\tau}}_{ij}^{(k)}  = \dfrac{\widehat{\tau}_{ij}^{(k)} - \dfrac{u^{(i)}-u^{(j)}}{n(n-1)}}{\sqrt{\left(1-\dfrac{U^{(i)}}{n(n-1)}\right)\left(1-\dfrac{U^{(j)}}{n(n-1)}\right)}}, \] де \[ U^{(i)} = \Sum_{g=1}^{m^{(i)}} n_g^{(i)} \left( n_g^{(i)} -1 \right), \] $m^{(i)}$ -- кількість груп об'єктів з нерозрізненними рангами по змінній $x^{(i)}$, $n_g^{(i)}$ -- кількість членів у $g$-й групі нерозрізнимих рангів по $i$-й змінній. \\

Коли коефіцієнт приймає проміжне значення, то перевіряємо гіпотезу $H_0: \widehat{\tau}_{ij}^{(s)} = 0$. Якщо об'єм вибірки невеликий, то перевіряємо по таблиці, при $n = 4..10$. Якщо ж $n > 10$, то використовуємо \[\left|\widehat{\tau}_{ij}^{(k)}\right|\le U_{\alpha/2} \sqrt{\dfrac{2(2n+5)}{9n(n-1)}}. \]
\subsubsection{Характеристика множинних рангових статистичних зв'язків}
Нехай аналізується $m$ змінних $\zeta = \left(x^{(k_1)}, x^{(k_2)}, \ldots, x^{(k_m)}\right)^*$ . \\

В якості характеристики використовується \textit{коефіцієнт конкордації}. \textit{Коефіцієнтом конкордації} для змінної $\zeta = \left(x^{(k_1)}, x^{(k_2)}, \ldots, x^{(k_m)}\right)^*$  називають величину \[ \widehat{w}_\zeta = \dfrac{12}{m^2(n^3-n)}\Sum_{i=1}^n \left(\left(\Sum_{j=1}^m x_i^{(j)}\right)-\dfrac{m(n+1)}{2}\right)^2. \]
\textbf{Властивості}:
\begin{enumerate}
	\item $0 \le \widehat{w}_\zeta \le 1$;
	\item якщо $\widehat{w}_\zeta = 1$, то ранжировки по змінним співпадають:
	\item якщо $\widehat{w}_\zeta = 0$, то відсутній зв'язок між ранжировками.
\end{enumerate}
У випадку двох нерозрізнених рангів використовуємо модифікований коефіцієнт $\widehat{w}_\zeta$:
\[ \widehat{\widehat{w}}_\zeta = \dfrac{\left(\left(\Sum_{j=1}^m x_i^{(k_j)}\right)-\dfrac{m(n+1)}{2}\right)^2}{\dfrac{m^2(n^3+n)}{2}-m \Sum_{j=1}^m T^{(k_i)}}, \] де \[ T^{(k_j)} = \dfrac{1}{12} \Sum_{g=1}^{m_j} \left(\left(n_g^{(k_j)}\right)^2-n_g^{(k_j)}\right). \]
Якщо $\widehat{w}_\zeta$ приймає проміжне значення, то робимо перевірку на значимість $H_0: \widehat{w}_\zeta = 0$, $0 < \alpha < 1$. Коли $n = 3..7$, $m = 2..20$, то за таблицею. Якщо $n > 7$, $m > 20$ , то розглядаємо статистику $\widehat{w}_\zeta$: \[\widehat{w}_\zeta < \dfrac{\chi_\alpha^2(n-1)}{m(n-1)},\] має $\chi^2$-розподіл з $(n-1)$ степенем свободи.
\subsection{Кореляційний аналіз номінальних змінних}
Нехай є змінна $\eta$ яка має $r_1$ градацій, та змінна $\xi$ яка має $r_2$ градацій:
\begin{table}[H]
	\centering
	\begin{tabular}{|c|c|c|c|c|c|}
	\hline
	& $1$ & $2$ & $\ldots$ & $r_1$ & $\sum$ \\
 \hline
	$1$ & $n_{11}$ & $n_{12}$ & $\ldots$ & $n_{1r_1}$ & $n_{1*}$ \\
 \hline
	$2$ & $n_{21}$ & $n_{22}$ & $\ldots$ & $n_{2r_1}$ & $n_{2*}$ \\
 \hline
	$\vdots$ & $\vdots$ & $\vdots$ & $\ddots$ & $\vdots$ & $\vdots$ \\
 \hline
	$r_2$ & $n_{r_21}$ & $n_{r_22}$ & $\ldots$ & $n_{r_2r_1}$ & $n_{r_2*}$ \\
 \hline
	$\sum$ & $n_{*1}$ & $n_{*2}$ & $\ldots$ & $n_{*r_1}$ & $n_{**}$ \\
 \hline
	\end{tabular}
\end{table}
Де $n_{ij}$ -- кількість таких спостережень $\{\eta=i,\xi=j\}$, позначимо $n_{i*} = \Sum_{j=1}^{r_1} n_{ij}$, $n_{*j} = \Sum_{i=1}^{r_2} n_{ij}$. \\

Вводимо статистику яка називається \textit{квадратичне спряження} і позначається \[ \chi_{\eta\xi}^2 = \Sum_{i=1}^{r_1} \Sum_{j=1}^{r_2} \dfrac{\left(n_{ij}-\dfrac{n_{i*}n_{*j}}{n}\right)^2}{\dfrac{n_{i*}n_{*j}}{n}}. \]
\textit{Коефіцієнти}:
\begin{enumerate}
	\item $\phi_{\eta\xi} = \sqrt{\dfrac{\chi_{\eta\xi}^2}{n}}$ -- \textit{середнє значення квадратичної спряженості};
	\item $P_{\eta\xi}=\sqrt{\dfrac{\chi_{\eta\xi}^2}{n+\chi_{\eta\xi}^2}}$ -- \textit{коефіцієнт Пірсона};
	\item $T_{\eta\xi}=\sqrt{\dfrac{\chi_{\eta\xi}^2}{n\sqrt{(r_1-1)(r_2-1)}}}$ -- \textit{коефіцієнт Чупрова};
	\item $T_{\eta\xi}=\sqrt{\dfrac{\chi_{\eta\xi}^2}{n\min(r_1-1,r_2-1)}}$ -- \textit{коефіцієнт Крамера}.
\end{enumerate}
\textbf{Властивості коефіцієнтів}
\begin{enumerate}
	\item $k_{\eta\xi}\ge0$, якщо коефіцієнт $p_{\eta\xi} \le 1$;
	\item $k_{\eta\xi}=0$, тоді зв'язок відсутній.
\end{enumerate}