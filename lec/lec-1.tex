\section{Вступ}
\subsection{Етапи аналізу даних:}
\begin{enumerate}
	\item Отримання і збереження даних.
	\item Обробка даних.
	\item Аналіз отриманих результатів (найважливіший етап).
\end{enumerate}
\subsection{Основні розділи аналізу даних:}
\begin{enumerate}
	\item Попередня обробка (включаючи розвідувальний аналіз) даних.
	\item Кореляційний аналіз -- застосування наявності зв'язків.
	\item Дисперсійний аналіз.
	\item Регресійний аналіз.
	\item Коваріаційний аналіз.
	\item Кластерний аналіз.
	\item Дискримінантний аналіз.
	\item Аналіз часових рядів.
\end{enumerate}
Пункти 3-5 -- побудова математичних моделей зв'язків.
\subsection{Класифікація змінних}
$\xi$, $\eta$, $\zeta$ -- змінні, які ми спостерігаємо. \\

$\{x_i\}_{i \in I}$, $\{y_j\}_{j \in J}$, $\{z_k\}_{k \in K}$ -- спостереження за змінними. \\

\textbf{Змінні}: кількісні і якісні.
\begin{itemize}
	\item Кількісні.
	\item Якісні:
	\begin{itemize}
		\item ординальні (порядкові);
		\item номінальні (класифікаційні).
	\end{itemize}
\end{itemize}
\textit{Ординальні змінні} -- змінні, що приймають значення з деякої множини, елементи якої називаються градаціями, причому кожен елемент множини апріорі впорядкований відносно інших (задано чіткий порядок) \\

\textbf{Приклад.} Рівень освіти: бакалавр, спеціаліст, магістр -- упорядковані змінні.\\

\textbf{Приклад.} Військові звання.\\

\textit{Номінальні змінні} -- змінні, що приймають своє значення з деякої множини, елементи (градації) якої не мають наперед заданого порядку (загальновідомого). \\

\textit{Категоризовані змінні} -- змінні, для яких апріорі відома множина їх значень (градацій) та алгоритм віднесення конкретного спостереження такими змінними до градації. \\

\textit{Некатегоризовані змінні} -- змінні, для яких апріорне задана або множина значень, або алгоритм віднесення спостереження до певної градації. \\

\textbf{Приклад.} Некатегоризовані змінні -- назви юр. осіб на даний момент. Влаштувались на роботу, зарплату не заплатили, фірма зникла, на іншій вулиці з'явилась $\Rightarrow$ градація зникла. \\

Ще існує поділ на дискретні і неперервні змінні.
\subsection{Групування даних}
Проводиться при спостереженнях над неперервними змінними (кількість спостережень $n > 50$). У дискретному випадку звертають увагу на кількість змінних $m > 10$. \\

\textbf{Ідея підходу:} вся вибірка спостережень розбивається на підвибірки і кожен заміняється на типового представника і далі працюють з цими представниками. \\

Нехай є вибірка. По ній знаходимо $\min$ і $\max$ значення.  \[ \{x_i\}_{i=1}^n: (x_{\min}, x_{\max}).\]
Цей інтервал розбиваємо на $s$ підінтервалів. Зазвичай $s$ вибирають так \[ 5 \le s \le 30 \quad \text{і} \quad s = 1 + \lfloor \log_2(n) \rfloor. \]
Беруть підінтервали $(C_1^1, C_1^2]$, $(C_2^1, C_2^2]$, $\ldots$, $(C_s^1, C_s^2)$. Потрібно, щоб в кожен інтервал потрапило більше 5 спостережень. Вибирають з кожного інтервалу єдиного представника. \\

Поставимо у відповідність $(C_i^1, C_i^2] \mapsto x_i^0$ (як правило середня точка середня точка), $v_i$ -- частота попадання. \\

Тобто переходимо від вибірки $\{x_i\}_{i=1}^n$ до вибірки $\{x_i^0, v_i\}_{i=1}^s$. \\

Зауваження: для випадкової величини $\xi$ -- $F_\xi(x)$ -- \textit{функція розподілу}. \\

$\widehat{F}_\xi(x)$ -- \textit{tемпіричний розподіл}, $n$ -- \textit{об'єм вибірки}. \\

$p_s(x)$, $\widehat{p}_s^{(n)}(x)$ -- \textit{неперервний випадок} (щільність). \\

Для дискретного випадку $\{y_i, p_i\}_{i=1}^m \mapsto \{ y_i, \widehat{p}_i\}_{i=1}^m$ -- \textit{полігон частот}.
\section{Розвідувальний аналіз}
Займається розробкою методів попереднього експрес аналізу інформації шляхом представлення її у вигляді таблиць або різного роду графічних зображень.
\subsection{Спостереження за однією змінною}
Засоби спостереження:
\begin{enumerate}
	\item пробіт-графік;
	\item імовірнісний графік;
	\item висячі гістобари;
	\item підвішена коренеграма;
	\item зображення ``скринька з вусами'';
	\item зображення ``стебло-листок''.
\end{enumerate}
У випадках 1-2 використовуємо інше зображення функцій розподілу, 3-4 -- використання іншого розподілу емпіричних функцій щільності, 5-6 -- сімейство розподілів зсув масштабу. \\

Сімейство розподілів $\mathcal{F}$ -- \textit{сімейство розподілів типу зсуву масштабу}, якщо існує функція розподілу \[ \exists F_0(\cdot) \in \mathcal{F}: \forall F(\cdot) \in \mathcal{F}: \exists a, b \in \RR^1, (b > 0): F(x) = F_0\left(\dfrac{x-a}{b}\right), \] де $a$ -- параметр зсуву, $b$ -- параметр масштабу, $F_0$ -- базова функція для сімейства розподілів $\mathcal{F}$. \\

\textbf{Приклад.} Нормальний розподіл $F(x): N(m, \sigma^2)$. Базова функція $\Phi(x)$ з розподілу $N(0,1)$. $a = m$, $b = \sigma$, $F(x) = \Phi\left(\frac{x - m}{\sigma}\right)$. Сімейство нормальних розподілів є сімейством зсуву масштабу. \\

\textbf{Приклад.} Експоненціальний розподіл з параметром $\lambda > 0$. $a = 0$, $b = \frac{1}{\lambda}$, $F(x) = \Phi_1(\lambda x)$, $\Phi_1$ -- базова функція експоненціального розподілу з параметром $\lambda = 1$. \\

\subsubsection{Пробіт-графік}
Будується наступним чином: \\

Маємо на вході вибірку $\{x_i\}_{i=1}^n$. \\

Обчислимо емпіричну функцію розподілу $\{x_i\}_{i=1}^n \mapsto \widehat{F}(x)$ (Сімейство розподілів $\mathcal{F}$ з базовою функцією $F_0$). \\

\textit{Пробіт-графік} -- графік функції $y = F_0^{-1}\left(\widehat{F}(x)\right)$. \\

\textbf{Використовується для:}
\begin{enumerate}
	\item Перевірки гіпотези $H_0: F_\xi(\cdot) \in \mathcal{F}$. \\

	У випадку, коли справедлива гіпотеза $H_0: F_\xi(\cdot) \in \mathcal{F}$. пробіт-графік повинен уявляти собою майже пряму. \\
 % figure 1
	Пояснення: маємо: \[ y = F_0^{-1}(\widehat{F}_\xi(x)) \overset{H_0}{\approx} F_0^{-1}\left(F_0\left(\dfrac{x-a}{b}\right)\right) = \dfrac{x}{b} - \dfrac{a}{b}. \]
	\item Виявлення наявності аномальних спостережень у вибірці. % figure 2
\end{enumerate}